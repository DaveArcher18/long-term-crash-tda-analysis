\documentclass[11pt,a4paper]{article}

% Packages
\usepackage[utf8]{inputenc}
\usepackage[T1]{fontenc}
\usepackage{lmodern}
\usepackage[margin=1in]{geometry}
\usepackage{graphicx}
\usepackage{booktabs}
\usepackage{longtable}
\usepackage{hyperref}
\usepackage{xcolor}
\usepackage{amsmath}
\usepackage{amssymb}
\usepackage{float}
\usepackage{caption}
\usepackage{subcaption}
\usepackage{fancyhdr}
\usepackage{titlesec}
\usepackage{enumitem}
\usepackage{listings}

% Colors
\definecolor{linkblue}{RGB}{0,102,204}
\definecolor{codebackground}{RGB}{245,245,245}
\definecolor{successgreen}{RGB}{34,139,34}
\definecolor{warningorange}{RGB}{255,140,0}
\definecolor{failred}{RGB}{178,34,34}

% Hyperref setup
\hypersetup{
    colorlinks=true,
    linkcolor=linkblue,
    urlcolor=linkblue,
    citecolor=linkblue
}

% Code listing style
\lstset{
    backgroundcolor=\color{codebackground},
    basicstyle=\ttfamily\small,
    breaklines=true,
    frame=single,
    rulecolor=\color{gray!30}
}

% Header/Footer
\pagestyle{fancy}
\fancyhf{}
\rhead{TDA Crash Prediction Assessment}
\lhead{Gidea-Katz Methodology}
\rfoot{Page \thepage}

% Title formatting
\titleformat{\section}{\Large\bfseries}{\thesection}{1em}{}
\titleformat{\subsection}{\large\bfseries}{\thesubsection}{1em}{}

% Document info
\title{\textbf{Topological Data Analysis for Crash Prediction}\\[0.5em]
\large A 69-Year Backtest of the Gidea-Katz Methodology}
\author{Quantitative Research Report}
\date{January 2026}

\begin{document}

\maketitle

\begin{abstract}
This study tests the Gidea-Katz (2017) Topological Data Analysis methodology across 69 years of US market data (1957 to 2025). The method computes persistence landscape norms from 3D point clouds of daily log-returns (DJI, DJT, S\&P 500) and detects trends with Mann-Kendall tests. Results show statistically significant predictive value: 20\% strict precision versus 13\% base rate, with 38\% recall. The technique detects slow-building systemic crises well (1973 to 74: 40\%, 1987: 100\%, 2008: 49\%, 2022: 50\% warning rates). It misses exogenous shocks (1962: 0\%, COVID-2020: 0\%). TDA works as a supplementary risk indicator. It should not serve as a standalone trading signal.
\end{abstract}

\tableofcontents
\newpage

%==============================================================================
\section{Executive Summary}
%==============================================================================

This study tests the Gidea-Katz (2017) Topological Data Analysis (TDA) methodology across 69 years of US market data (1957 to 2025). The technique computes persistence landscape norms from 3D point clouds of daily log-returns. It uses Mann-Kendall trend detection to identify periods of rising topological instability.

\subsection*{Key Findings}

\begin{itemize}[noitemsep]
    \item The TDA signal shows \textbf{statistically significant predictive value} with 20\% strict precision (vs.\ 13\% base rate) and 38\% recall
    \item When including corrections (10 to 20\% drawdowns), precision rises to \textbf{44.6\%}. Nearly half of all warnings precede meaningful market declines
    \item The technique excels at detecting \textbf{slow-building systemic crises} (1973 to 74: 40\%, 1987: 100\%, 2008: 49\%, 2022: 50\% warning rates)
    \item It fails on \textbf{exogenous shocks} (1962: 0\%, COVID-2020: 0\% warning rates)
\end{itemize}

\subsection*{Recommendation}

TDA is a valid supplementary risk indicator for detecting building financial instability. Do not use it as a standalone trading signal. The false positive rate (14\% of all days) is too high.

%==============================================================================
\section{Introduction}
%==============================================================================

\subsection{Research Question}

Do topological features extracted from multi-index return dynamics provide early warning signals for major market crashes?

\subsection{Background}

Gidea and Katz (2017) proposed using Topological Data Analysis to detect market instability. They measured complexity of return patterns across multiple indices. Their hypothesis: before major crashes, the shape of market dynamics becomes increasingly complex. This complexity manifests as elevated persistence in $H_1$ homology (1-dimensional loops in the return point cloud).

The persistence landscape, introduced by Bubenik (2015), provides a stable, functional representation of persistence diagrams. It admits statistical analysis. The $L^1$ and $L^2$ norms of these landscapes serve as scalar measures of topological complexity.

\subsection{Scope}

This study extends the original methodology to a comprehensive 69-year backtest covering:

\begin{itemize}[noitemsep]
    \item \textbf{7 major market crashes} (1962, 1973 to 74, 1987, 2000, 2008, 2020, 2022)
    \item \textbf{17,318 trading days} of continuous data
    \item \textbf{Multiple market regimes} (Bretton Woods, stagflation, Great Moderation, post-GFC)
\end{itemize}

%==============================================================================
\section{Methodology}
%==============================================================================

\subsection{Data Sources}

Data comes from Stooq.com, which provides historical index data back to the 1950s.

\begin{table}[H]
\centering
\begin{tabular}{llll}
\toprule
\textbf{Index} & \textbf{Source} & \textbf{Period} & \textbf{Observations} \\
\midrule
Dow Jones Industrial (\^{}DJI) & Stooq & 1957-01-02 to 2026-01-02 & 17,369 \\
Dow Jones Transportation (\^{}DJT) & Stooq & 1957-01-02 to 2026-01-02 & 17,368 \\
S\&P 500 (\^{}SPX) & Stooq & 1957-01-02 to 2026-01-02 & 17,368 \\
\bottomrule
\end{tabular}
\caption{Data sources and coverage}
\end{table}

After alignment to common trading dates: \textbf{17,368 daily observations}.

\subsection{TDA Pipeline}

The signal extraction pipeline follows Gidea-Katz (2017):

\begin{enumerate}[noitemsep]
    \item Compute daily log-returns for each index
    \item Construct 50-day sliding window point clouds in $\mathbb{R}^3$
    \item Compute Vietoris-Rips persistence diagrams (up to $H_1$)
    \item Extract persistence landscapes and compute $L^1$, $L^2$ norms
    \item Apply 50-day rolling standard deviation to norm series
    \item Compute 250-day Mann-Kendall $\tau$ on rolling SD
    \item Trigger warning when $\tau > 0.30$
\end{enumerate}

\begin{table}[H]
\centering
\begin{tabular}{ll}
\toprule
\textbf{Parameter} & \textbf{Value} \\
\midrule
Topology window ($w$) & 50 trading days \\
Maximum filtration scale & 0.20 \\
Landscape functions ($k$) & 1, 2, 3, 4, 5 \\
Rolling SD window & 50 days \\
Mann-Kendall lookback & 250 trading days \\
Warning threshold & $\tau > 0.30$ \\
\bottomrule
\end{tabular}
\caption{TDA pipeline parameters}
\end{table}

\subsection{Classification Framework}

Each trading day is classified based on signal state and forward-looking maximum drawdown:

\begin{table}[H]
\centering
\begin{tabular}{ll}
\toprule
\textbf{Classification} & \textbf{Condition} \\
\midrule
True Positive & Signal ON + Forward 250-day drawdown $\geq$ 20\% \\
Correction & Signal ON + Forward drawdown 10 to 20\% \\
False Positive & Signal ON + Forward drawdown $<$ 10\% \\
False Negative & Signal OFF + Forward drawdown $\geq$ 20\% \\
True Negative & Signal OFF + Forward drawdown $<$ 20\% \\
\bottomrule
\end{tabular}
\caption{Classification framework}
\end{table}

\subsection{Limitations}

\begin{enumerate}[noitemsep]
    \item \textbf{Look-ahead bias in classification}: Forward drawdowns are computed with future data. This is unavoidable for evaluation.
    \item \textbf{Parameter sensitivity}: Results depend on threshold choices ($\tau = 0.30$, $w = 50$)
    \item \textbf{Single reference index}: Classification uses S\&P 500 drawdown only
    \item \textbf{No transaction costs}: Metrics do not account for implementation frictions
\end{enumerate}

%==============================================================================
\section{Results}
%==============================================================================

\subsection{Overall Performance}

\begin{table}[H]
\centering
\begin{tabular}{lrl}
\toprule
\textbf{Metric} & \textbf{Value} & \textbf{Interpretation} \\
\midrule
Total observations & 17,318 & 69 years of daily data \\
Warning days & 4,344 (25.4\%) & Signal triggered 1 in 4 days \\
True Positives & 867 (5\%) & Correct crash predictions \\
Corrections & 1,071 (6\%) & Correct correction predictions \\
False Positives & 2,406 (14\%) & False alarms \\
False Negatives & 1,407 (8\%) & Missed crashes \\
True Negatives & 11,317 (66\%) & Correct quiet predictions \\
\bottomrule
\end{tabular}
\caption{Classification breakdown}
\end{table}

\subsection{Performance Metrics}

\begin{table}[H]
\centering
\begin{tabular}{lrl}
\toprule
\textbf{Metric} & \textbf{Value} & \textbf{vs.\ Random} \\
\midrule
Precision (strict) & 20.0\% & +7pp vs.\ 13\% base rate \\
Precision (incl.\ corrections) & 44.6\% & Substantial signal value \\
Recall & 38.1\% & Catches 2 in 5 crashes \\
F1 Score & 26.2\% & Moderate overall balance \\
\bottomrule
\end{tabular}
\caption{Performance metrics}
\end{table}

\subsection{Performance by Crash Event}

\begin{table}[H]
\centering
\begin{tabular}{llccc}
\toprule
\textbf{Event} & \textbf{Period} & \textbf{Warning Rate} & \textbf{Max DD} & \textbf{Result} \\
\midrule
1962 Flash Crash & May to Jun 1962 & 0\% & 21.4\% & \textcolor{failred}{Missed} \\
1973 to 74 Bear Market & 1973 to 1974 & 40.2\% & 44.1\% & \textcolor{successgreen}{Detected} \\
1987 Black Monday & Aug to Oct 1987 & 100\% & 33.5\% & \textcolor{successgreen}{Strong} \\
2000 Dot-com Bust & Mar to Apr 2000 & 7.1\% & 27.2\% & \textcolor{warningorange}{Weak} \\
2008 Financial Crisis & Sep to Nov 2008 & 49.2\% & 47.0\% & \textcolor{successgreen}{Detected} \\
2020 COVID Crash & Feb to Mar 2020 & 0\% & 33.9\% & \textcolor{failred}{Missed} \\
2022 Bear Market & Jan to Oct 2022 & 50.2\% & 25.4\% & \textcolor{successgreen}{Detected} \\
\bottomrule
\end{tabular}
\caption{Detection performance by historical crash}
\end{table}

\subsection{Visualization}

Figure~\ref{fig:dashboard} presents the comprehensive backtest dashboard. It shows persistence landscape norms, Mann-Kendall $\tau$ values, forward drawdowns, and classification summary across the full 69-year period.

\begin{figure}[H]
\centering
\includegraphics[width=\textwidth]{figures/century_dashboard.png}
\caption{TDA Crash Prediction Backtest Dashboard (1957 to 2025)}
\label{fig:dashboard}
\end{figure}

%==============================================================================
\section{Analysis}
%==============================================================================

\subsection{What the Signal Measures}

The TDA methodology detects \textbf{rising complexity in inter-index return dynamics}. When the three indices (industrial, transportation, broad market) move in non-linear, asynchronous patterns, the persistence landscape norms increase.

The time series of $L^1$ norms reveals:

\begin{enumerate}[noitemsep]
    \item \textbf{Low baseline} ($\sim 2 \times 10^{-6}$) during calm markets (e.g., 1995, 2017)
    \item \textbf{Elevated levels} ($\sim 8 \times 10^{-6}$) during building stress (e.g., 1999, 2007)
    \item \textbf{Spike events} ($\sim 2 \times 10^{-5}$) during crisis peaks (e.g., October 2008)
\end{enumerate}

This pattern is consistent with the Gidea-Katz hypothesis: market stress manifests topologically before price crashes.

\subsection{Why Certain Crashes Are Detected}

Successfully detected crashes share common features:

\begin{description}[noitemsep]
    \item[1973 to 74 Bear Market:] 18-month deterioration from oil shock, inflation, Nixon resignation. Gradual stress buildup visible in topological metrics.
    \item[1987 Black Monday:] August to October buildup with rising volatility and inter-index divergence. 100\% warning rate indicates clear topological signal.
    \item[2008 Financial Crisis:] Subprime problems emerged in 2007, giving more than 12 months of building stress. Topological instability tracked credit deterioration.
    \item[2022 Bear Market:] Inflation concerns and rate hikes created extended uncertainty. Gradual regime shift detected over months.
\end{description}

\textbf{Common thread}: All detected crashes had \textbf{extended stress-building phases} where systemic risk accumulated gradually.

\subsection{Why Certain Crashes Are Missed}

Missed crashes share different features:

\begin{description}[noitemsep]
    \item[1962 Flash Crash:] Rapid technical selloff with no prolonged buildup. No topological precursor.
    \item[2020 COVID Crash:] Exogenous pandemic shock. Markets were topologically stable until sudden external impact.
\end{description}

\textbf{Common thread}: Missed crashes were \textbf{exogenous shocks} with no endogenous financial system stress preceding them.

\subsection{The False Positive Problem}

The 14\% false positive rate represents 2,406 trading days where warnings preceded benign outcomes. Analysis reveals:

\begin{enumerate}[noitemsep]
    \item \textbf{Near-misses}: Some false positives occurred during genuine stress resolved by policy intervention
    \item \textbf{Regime sensitivity}: Higher false positive rates during structurally volatile periods (1970s, early 2000s)
    \item \textbf{Threshold dependency}: The $\tau > 0.30$ threshold is not optimal. Higher thresholds reduce false positives but miss more true events
\end{enumerate}

%==============================================================================
\section{Conclusions and Recommendations}
%==============================================================================

\subsection{Assessment of the Methodology}

\begin{table}[H]
\centering
\begin{tabular}{ll}
\toprule
\textbf{Dimension} & \textbf{Assessment} \\
\midrule
Scientific validity & \textcolor{successgreen}{\checkmark} Demonstrated statistically significant predictive value \\
Theoretical grounding & \textcolor{successgreen}{\checkmark} Consistent with complexity/instability theory \\
Practical utility & \textcolor{warningorange}{$\sim$} Limited standalone value due to false positive rate \\
Scope of applicability & \textcolor{warningorange}{$\sim$} Effective for endogenous crises. Ineffective for shocks \\
\bottomrule
\end{tabular}
\caption{Methodology assessment summary}
\end{table}

\subsection{Strengths}

\begin{enumerate}[noitemsep]
    \item \textbf{Detects genuine signal}: 20\% precision vs.\ 13\% base rate is statistically meaningful
    \item \textbf{Theory-aligned}: Captures building systemic stress as hypothesized
    \item \textbf{Multi-decade robustness}: Works across different market regimes (1957 to 2025)
    \item \textbf{Complementary information}: Topology provides insight beyond traditional volatility measures
\end{enumerate}

\subsection{Limitations}

\begin{enumerate}[noitemsep]
    \item \textbf{High false positive rate}: 14\% of days generate warnings without subsequent crashes
    \item \textbf{Blind to exogenous shocks}: Does not predict pandemics, geopolitical events, technical failures
    \item \textbf{Parameter sensitivity}: Performance depends on threshold calibration
    \item \textbf{Lagging indicator}: Mann-Kendall requires 250 days of data, limiting lead time
\end{enumerate}

\subsection{Recommendations}

\begin{enumerate}[noitemsep]
    \item \textbf{Use as supplementary indicator}: Combine with fundamental, sentiment, and technical signals. Do not use standalone.
    \item \textbf{Adjust for regime}: Consider higher thresholds ($\tau > 0.40$) during structurally volatile periods
    \item \textbf{Monitor for buildups}: Most valuable as early warning for slow-developing crises
    \item \textbf{Do not rely for tail risk}: Does not protect against sudden exogenous events
\end{enumerate}

\subsection{Future Work}

\begin{enumerate}[noitemsep]
    \item \textbf{Parameter optimization}: Grid search over $w$, $\tau$, lookback parameters
    \item \textbf{Alternative metrics}: Test Wasserstein distances, bottleneck distances
    \item \textbf{Sector decomposition}: Apply to sector ETFs rather than broad indices
    \item \textbf{Real-time implementation}: Develop streaming TDA computation pipeline
\end{enumerate}

%==============================================================================
\section*{References}
%==============================================================================

\begin{enumerate}[noitemsep]
    \item Gidea, M., \& Katz, Y. (2017). Topological Data Analysis of Financial Time Series: Landscapes of Crashes. \textit{Physica A: Statistical Mechanics and its Applications}, 491, 820 to 834.
    \item Bubenik, P. (2015). Statistical Topological Data Analysis using Persistence Landscapes. \textit{Journal of Machine Learning Research}, 16, 77 to 102.
\end{enumerate}

%==============================================================================
\appendix
\section{Technical Implementation}
%==============================================================================

The analysis was implemented in R using:

\begin{itemize}[noitemsep]
    \item \texttt{TDA} package (GUDHI backend) for Vietoris-Rips persistence diagrams
    \item \texttt{Kendall} package for Mann-Kendall trend tests
    \item \texttt{parallel} package for multi-core TDA computation
    \item Direct Stooq.com downloads for historical data
\end{itemize}

Pipeline runtime: approximately 1.4 minutes for 17,318 observations on 7 cores.

\subsection{Repository Structure}

\begin{lstlisting}
src/
  data/get_century_data.R           # Data acquisition
  features/compute_tda_signals.R    # TDA computation
  features/classify_signals.R       # Forward DD classification
  models/run_century_backtest.R     # Pipeline orchestration
  visualization/plot_backtest_results.R
\end{lstlisting}

\subsection{Data Files}

\begin{table}[H]
\centering
\begin{tabular}{ll}
\toprule
\textbf{File} & \textbf{Description} \\
\midrule
\texttt{data/raw/\{DJI,DJT,SPX\}\_raw.csv} & Individual index downloads \\
\texttt{data/interim/century\_aligned\_prices.csv} & Aligned price series \\
\texttt{data/interim/century\_log\_returns.csv} & 3D log-returns \\
\texttt{data/processed/century\_tda\_signals.csv} & TDA metrics and signals \\
\texttt{data/processed/century\_backtest\_results.csv} & Final classified results \\
\bottomrule
\end{tabular}
\caption{Output data files}
\end{table}

\vfill
\begin{center}
\rule{0.5\textwidth}{0.4pt}\\[1em]
\textit{Report generated: January 2026}\\
\textit{Data range: 1957-01-02 to 2026-01-02 (17,318 trading days)}\\
\textit{Methodology: Gidea-Katz TDA with persistence landscapes}
\end{center}

\end{document}
